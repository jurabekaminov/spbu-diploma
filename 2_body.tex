\chapter{Введение}
    \section{Актуальность работы}

    За последние 50 лет сельскохозяйственная отрасль радикально изменилась. Достижения в области машиностроения увеличили масштабы, скорость и производительность сельскохозяйственной техники, что привело к более эффективной обработке большего количества земель. Семена, ирригация и удобрения также значительно улучшились, что помогло фермерам повысить урожайность. Сейчас сельское хозяйство находится на заре ещё одной революции, в основе которой лежат информационные технологии(ИТ). 
    
    Одним из способов использования ИТ в сельском хозяйстве является точное земледелие. Точное земледелие — комплексная высокотехнологичная система сельскохозяйственного менеджмента, цель которой - оптимизация затрат и получение максимальной прибыли и урожая. Сама концепция возникла ещё в 1980-х годах, когда в США стали делать первые карты по дифференцированному внесению удобрений на основании анализов почвы. \cite{iowa} Но широкое распространение идеи точного земледелия получили в последние пять лет благодаря развитию мобильных технологий, высокоскоростному интернету и точным спутниковым данным. \cite{wiki1, onesoil_blog}
    
    Точное земледелие включает множество отдельных технологий, необходимость внедрения которых определяется на усмотрение собственников и руководителей агропредприятия. Примерами таких технологий являются: глобальное спутниковое позиционирование (GPS), оценка урожайности YMT (Yield Monitor Technologies), геоинформационные системы (GIS), дистанционное зондирование земли (ДЗЗ) и другие.  Можно задействовать как все технологии, входящие в комплекс, или только некоторые, эффект от которых будет наиболее значимым для данного предприятия. \cite{kursk, bis}



    \section{Практическая значимость работы}
    
    Серверная часть веб-приложения будет интегрирована с клиентской частью с целью получения готового продукта для конечного пользователя.

    Пользователями данного веб-сервиса могут быть организации и фермеры, владеющие сельхозугодьями, агрономы, учёные и другие заинтересованные лица, являющиеся сотрудниками соответствующих организаций или фермеров. <<Цифровой двойник сельскохозяйственного поля>> позволит им провести цифровизацию различных процессов управления сельскохозяйственными полями. 
    
    Ключевыми же факторами для цифровизации данных процессов являются оптимизация использования ресурсов, таких как химические удобрения и пестициды, а также максимизация урожайности. Эти факторы могут быть достигнуты за счёт сбора и анализа метеоданных, данных о почве, аэрофотоснимков и т.д. 
    
    Так, метеоданные, такие как температура и влажность могут оптимизировать принятие решений по планированию будущих севооборотов на конкретных полях.
    
    В то же время использование агрохимических данных может способствовать идентификации областей с высоким уровнем риска возникновения болезней, плесени и других проблем, которые могут негативно сказаться на урожае. Это позволяет предпринимать своевременные меры для предотвращения потерь урожая и сокращения расходов на лечение и защиту растений.

    
    
    \section{Цель работы} \label{target}
    
    Данная работа посвящена разработке серверной части веб-приложения <<Цифровой двойник сельскохозяйственного поля>> - геоинформационной системы для точного земледелия. Процесс разработки можно разбить на следующие части:

    \begin{enumerate}
            \item Обзор исследований и существующих решений;
            \item Формирование требований к приложению;
            \item Проектирование архитектуры серверной части приложения;
            \item Программная разработка серверной части приложения;
            \item Тестирование и развертывание приложения на удалённом сервере;

    \end{enumerate}


    
\chapter{Обзор исследований и существующих решений}
    \section{Обзор исследований}

    В статье \cite{elsevier} рассматривается  <<Soil daTA Retrival Tool (START)>> - инструмент подготовки данных для имитационных моделей сельскохозяйственных культур с использованием WEB-ориентированных баз данных о почвах. На примере <<START>> утверждается, что создание и инициализация входных файлов для почвенных данных занимают около 0.33 \% от времени, которое требуется на подготовку данных вручную. Эти результаты позволяют предположить, что использование специализированного ПО может обеспечить эффективный подход к подготовке файлов входных данных о почвах. 
    
    В работе \cite{busel} были выполнены исследования в рамках развития базовых технологий точного земледелия для улучшения производственных и экологических показателей растениеводства. Эти исследования были направлены на определение оптимальной урожайности на основе климатических и почвенных факторов, а также на разработку методов управления качеством и объёмом урожая. Был сделан вывод, что применение дифференцированных агротехнологий наиболее перспективно для эффективного управления производством растениеводческой продукции. 

    Исследования \cite{aero1}, \cite{aero2} указывают на широкое применение аэрофотоснимков в точном земледелии с акцентом на детальном дифференцированном подходе к управлению системой "поле-посев". Новые методы, основанные на анализе аэрофотоснимков, представляют собой перспективную альтернативу традиционным методам полевого анализа уровня питательности растений и необходимости использования химических средств защиты. Эти методы показали значительные преимущества, включая высокую разрешающую способность снимков, быстрое и экономичное получение изображений и результатов, а также возможность наблюдения за посевами в различные временные интервалы.
    
    В исследовании \cite{service} авторы разрабатывают сервис-ориентированную архитектуру и интегрируют пространственные данные для агроконсультационных систем на основании экспериментов, проведённых в Северном Гуджарате.
    Результатом данной работы является архитектура с \hyperref[REST]{RESTful} веб-сервисами, обрабатывающими запросы фермеров для манипуляции геопространственными данными, хранящимися в PostgreSQL. В заключение авторы утверждают, что существует необходимость обобщить предложенную ими архитектуру, а также необходимо протестировать и реализовать дальнейшее бесшовное взаимодействие между источниками пространственных данных.
    

    \section{Обзор существующих решений} \label{existing}

    На сегодняшний день существует множество компьютерных программ для решения задач точного земледелия. Рассмотрим самые популярные из них. 

    \begin{enumerate}
       
    
         \item \textit{Агроаналитика-IoT} \cite{smartagro} - онлайн-сервис оказания консультационных услуг , в области сельскохозяйственного производства с целью снижения неопределённости и, соответственно, рисков в агропромышленном комплексе; создания благоприятного инвестиционного климата; увеличения рентабельности производства сельскохозяйственной продукции. 
         
        \textbf{Плюсы}:  
        \begin{itemize}
                \item Широкий функционал: автоматизация и прозрачность бизнес-процессов агропредприятия, планирование агрономических и инженерных работ, кадастровый контроль и работа с земельным фондом и т.п.;
                \item Высокая производительность IOT сервера: получение треков техники по 1000 единиц за сутки занимает около 10 секунд, получение последнего местоположения - 5 секунд;
                \item Возможность интеграции с другими системами (например, 1C);
                \item Наличие чат-бота и мобильного приложения; 
        \end{itemize}
        \textbf{Минусы}:
        \begin{itemize}
                \item Нет поддержки функционала для работы с аэрофотоснимками;
                \item Цена: от 30р в год за гектар сельхозугодий. Доступны разные тарифы;
        \end{itemize}


        \item \textit{Cropwise Operations}\cite{cropwise} - это система спутникового мониторинга полей, специально разработанная для руководителей аграрных компаний и агрономов. Система контролирует состояние посевов в режиме реального времени, следит за вегетацией полей в различных регионах, выявляет проблемные участки и рассчитывает рекомендованную норму удобрений.
         
        \textbf{Плюсы}:  
        \begin{itemize}
                \item Широкий функционал: мониторинг и контроль посевных площадей, уточненный прогноз погоды для каждого поля, создание заданий по выполнению работ на поле и т.п.;
                \item Возможность интеграции с другими системами (например, 1C);
                \item Автоматические оповещения о превышении скорости, выезде за пределы геозоны, сливе топлива;
                \item Наличие мобильного приложения; 
        \end{itemize}
        \textbf{Минусы}:
        \begin{itemize}
                \item Нет поддержки функционала для работы с аэрофотоснимками;
                \item Цена: от \$1 до \$5 за гектар в год, в зависимости от страны и размера полей, которые вы собираетесь контролировать;
        \end{itemize}


    \item \textit{OneSoil} \cite{onesoil} - платформа, которая помогает эффективно управлять полями. Платформа объединяет мобильное приложение OneSoil Scouting, веб-приложение OneSoil и инструменты PRO-версии для точного земледелия OneSoil Yield. Вместе они помогают отслеживать изменения на полях, планировать сельскохозяйственные работы, повышать урожайность поля и экономить ресурсы вашей фермы.
         
        \textbf{Плюсы}:  
        \begin{itemize}
                \item Широкий функционал: Создание полей, внесение севооборотов, возможность делиться полями и заметками, получение метеоданных и т.д.; 
                \item Наличие мобильного приложения;
                \item В PRO-версии доступны инструменты для дифференцированного посева;
                \item В PRO-версии доступна интеграция с другими платформами;
                \item Наличие мобильного приложения; 
        \end{itemize}
        \textbf{Минусы}:
        \begin{itemize}
                \item Нет поддержки функционала для работы с аэрофотоснимками;
                \item Некоторые функции доступны только в платной PRO-версии;
        \end{itemize}



      \item \textit{QGIS}\cite{qgis} - бесплатное ПО для геоинформационного анализа. Оно используется для создания и редактирования геоданных. QGIS широко используется географами, картографами, специалистами по геодезии и геоинформатике. Еще одним применением QGIS может быть мониторинг и анализ сельхоз полей. Важно отметить, что QGIS сам по себе не является готовым продуктом <<из коробки>> в сравнении с предыдущими решениями, а требует самостоятельного написания кода и настройки под свои задачи. 
         
        \textbf{Плюсы}:  
        \begin{itemize}
                \item Гибкость в настройке: пользователи могут настраивать интерфейс, разрабатывать и настраивать собственные инструменты на языке программирования Python;
                \item Открытое API: как следствие, возможность интеграции почти с любыми системами;
                \item Цена: бесплатно;
        \end{itemize}
        \textbf{Минусы}:
        \begin{itemize}
                \item Не является готовым продуктом <<из коробки>>. Требует самостоятельного написания кода и настройки под свои задачи.
                \item Cпециалистам сельскохозяйственной сферы сложно использовать этот инструмент из-за отсутствия экспертных знаний.
                
        \end{itemize}
    \end{enumerate}

    
    В процессе исследования были рассмотрены и другие решения: АгроСигнал \cite{agrosignal}, АгроМон \cite{agromon} и другие. Все исследуемые решения, за исключением QGIS имеют богатый функционал, отличную техническую поддержку и сопровождение. Однако за подключение к системе и обслуживание взимается плата. 

    QGIS, напротив, является бесплатным ПО с открытым исходным кодом, но требует самостоятельной настройки и написания кода. QGIS является единственным рассмотренным решением с поддержкой функционала для работы с аэрофотоснимками.


\chapter{Формирование требований к приложению}
    \section{Функциональные требования} \label{functional}
    
    \begin{enumerate}
        
        \item Должна быть спроектирована система хранения и обработки информации о сельхозугодье (поле). Так, для каждого поля должны храниться:
            \begin{itemize}
                \item название, площадь, полигон и контур поля;
                \item данные о текущем посеве: дата начала, дата окончания, культура и т.д. 
             \end{itemize}
             Посевы полей могут меняться со временем, но должна храниться история всех посевов для каждого поля - севооборот;

        \item Должны быть разработаны механизмы аутентификации и авторизации, дающие защиту от несанкционированного доступа к данным и управления ресурсами. Так, пользователями приложения могут быть:
            \begin{itemize}
                \item организации: предприятия/фермеры являющиеся владельцами сельхозугодий, они могут предоставлять доступ к полям для своих сотрудников;
                \item сотрудники: наблюдатели, инженеры, трактористы, и т.д. Они обладают меньшими правами доступа по сравнению с организациями и фермерами;
            \end{itemize}
            Регистрацией корпоративного аккаунта сотрудника в системе занимается организация на которую работает данный сотрудник.
        
        \item Пользователи должны иметь профили, содержащие их личные данные.
            Профиль организации состоит из следующих данных:
            \begin{itemize}
                \item название
                \item описание (опционально)
                \item город
                \item ИНН (опционально)
                \item номер телефона (опционально)
                \item веб-сайт (опционально)
            \end{itemize}
            
            Профиль сотрудника состоит из следующих данных:
            \begin{itemize}
                \item имя
                \item фамилия
                \item отчество (опционально)
                \item дата рождения (опционально)
                \item номер телефона (опционально)
            \end{itemize}
            
        \item Должна быть разработана ETL-система хранения, сбора, обработки метеоданных о полях, выбранных пользователем.
        
            Каждый день в 5:00, 9:00, 13: 00 и 17:00 должны загружаться следующие данные, относящиеся к сельхозугодью, актуальные на момент их получения:
            \begin{itemize}
                \item температура
                \item влажность
                \item уровень осадков (мм)
                \item точка росы
                \item температура почвы на глубине/глубинах 0-18 см
                \item влажность почвы на глубине/глубинах от 0-27 см
            \end{itemize}
            
            Также раз в сутки должны загружаться следующие данные:
            \begin{itemize}
                \item максимальная и минимальная температуры
                \item время восхода солнца
                \item время заката
                \item сумма осадков (мм)
            \end{itemize}

            
    \end{enumerate}
            
    
    
    \section{Нефункциональные требования} \label{non_functional}
    
    Как мы могли видеть из \ref{existing}, коммерческие приложения для точного земледелия, в основном, имеют широкий функционал. Следовательно, приложение должно быть открыто к добавлению нового функционала, интеграции с другими системами. 
    
\chapter{Проектирование архитектуры серверной части приложения}
    \section{Выбор паттерна проектирования серверной части приложения} \label{pattern}

    Паттерны проектирования - это описания коммуникационных объектов и классов, которые настроены для решения общей проблемы проектирования в определенном контексте. \cite{design}
    
    На сегодняшний день существуют два популярных подхода к проектированию веб-приложений: монолитное приложение и микросервисы.
        \subsection{Монолитная архитектура}
        \textbf{Монолитная архитектура} — это классический подход, который используется при создании программных приложений. Являлся доминирующим подходом к разработке программного обеспечения на протяжении многих лет и до сих пор широко используется. Монолитная архитектура объединяет и запускает все компоненты приложения как единую унифицированную систему. Таким образом, любые изменения, внесенные в одну часть приложения, потенциально могут повлиять на всю систему. \cite{baeldung}
    
        На рисунке 1 мы можем видеть, что слои пользовательского интерфейса, бизнес-логики и манипулирования данными объединены в одно приложение. 
    
        \noindent\begin{minipage}[t]{\textwidth}
             \begin{center}
             \includegraphics[width=0.8 \textwidth]{images/mono.png}
             \captionof{figure}{Монолитная архитектура. Источник: \cite{baeldung}}\end{center} \label{fig:Moreno}
        \end{minipage}

        Выделим ключевые преимущества монолитной архитектуры:

        \begin{itemize}
            \item \textbf{Упрощенная разработка, тестирование и развертывание:} Команды создают и тестируют монолитные приложения в согласованной инфраструктуре с использованием аналогичных инструментов в виде единой базы кода, а затем развертывают эти приложения как один контейнер в рабочей среде.
            \item \textbf{Быстрое решение проблем:} Несмотря на то, что компоненты приложения в монолитной архитектуре более сложны, чем в микросервисах, запросы рабочих процессов легче отслеживать, что позволяет разработчикам обнаруживать проблемы по мере их возникновения. \cite{monovsmicro}
            
        \end{itemize}

         Ключевые недостатки:
          \begin{itemize}
            \item \textbf{Трудности масштабирования:} монолитные приложения имеют тесно связанные программные компоненты, работа которых зависит друг от друга. Эффективное масштабирование практически невозможно. 
            \item \textbf{Сильная связность:} когда один компонент системы замедляется или выходит из строя, все приложение может перестать работать должным образом.
            \item \textbf{Трудноcти поддержки:} по мере роста кодовой базы становится сложнее поддерживать приложение, тратится много времени на ускорение работы ресурсов и устранение неполадок. \cite{baeldung, monovsmicro}

            
            
        \end{itemize}

        Монолитная архитектура является оптимальным решением когда известно, что размер и основной функционал приложения не будет расширяться. Разработчики могут работать над одним и тем же кодом. Для разработки монолитного приложения требуется меньше разработчиков, чем для разработки микросервисного. 
        
        \subsection{Микросервисная архитектура}
        \textbf{Микросервисная архитектура} представляет собой другую методологию разработки программных приложений по сравнению с монолитной. Так, приложение разбивается на небольшие независимые сервисы, которые взаимодействуют друг с другом с помощью четко определенных API(application programming interface). Более того, каждый микросервис отвечает за определенную функциональность приложения. Каждый сервис можно разрабатывать, развертывать и масштабировать независимо от других сервисов. Таким образом, обеспечивается большая гибкость при разработке и поддержке приложения. \cite{baeldung, monovsmicro}
        
        Пример микросервисной архитектуры, можно увидеть на рисунке 2.
    
        \noindent\begin{minipage}[t]{\textwidth}
             \begin{center}
             \includegraphics[width=0.8 \textwidth]{images/micro.png}
             \captionof{figure}{Микросервисная архитектура. Источник: \cite{baeldung}}\end{center} \label{fig:Moreno}
        \end{minipage}

    
    Выделим ключевые преимущества микросервисной архитектуры:

        \begin{itemize}
            \item \textbf{Масштабируемость:} c помощью микросервисов команды могут разрабатывать независимые сервисы, ориентированные на конкретную проблему. Как правило, кодовая база микросервисов не сильно разрастается, что делает поддержку микросервисов более простым процессом.
            \item \textbf{Эффективное использование ресурсов и изолированность:} поскольку микросервисы независимы, организации могут масштабировать отдельные микросервисы в удобном для них темпе без необходимости масштабирования всего приложения. Команды также имеют свободу выбирать собственные технологические стеки для создания микросервисов.
            \item \textbf{Отказоустойчивость:} Если какой-либо из микросервисов перестанет работать, то это не повлечет за собой полную остановку работы всей системы. Достаточно будет устранить неполадку в неработающем микросервисе. \cite{monovsmicro}
            
        \end{itemize}

         Ключевые недостатки:
          \begin{itemize}
            \item \textbf{Сложность развертывания:} чем больше микросервисов, тем сложнее процесс их развертывания на сервере.
            \item \textbf{Необходимо больше ресурсов:} для достижения успеха отдельным командам необходимы собственные системы, программное обеспечение, инфраструктура и операционные процессы, что может привести к более высоким затратам как на оборудование, так и на зарплаты разработчикам. 
            \item \textbf{Долгий старт:} Для разработки MVP(минимального жизнеспособного продукта) требуется разработать больше компонент, чем требовалось бы для монолитного приложения.
        \end{itemize}

       Микросервисная архитектура предоставляет свободу выбора инструментов для разработки: команды могут использовать те инструменты, которые им нужны для добавления нового функционала. Также, повышается надежность и отказоустойчивость системы, так как микросервисы представляют собой независимые приложения, и в случае возникновения неисправности у одного из них, система не перестанет функционировать. \cite{monovsmicro}
            
       Так как основное нефункциональное требование к серверной части приложения - возможность масштабирования \ref{non_functional}, то микросервисная архитектура будет оптимальным выбором для данной задачи.




        \section{Межсервисное общение} \label{messaging}
        После того как мы определились с выбором микросервисной архитектуры, надо определиться с тем, каким образом микросервисы будут взаимодействовать в системе. Общение клиента с микросервисами в микросервисной архитектуре обычно осуществляется через сетевые протоколы и API. Далее, рассмотрим некоторые распространённые способы общения клиента с микросервисами.

        \subsection{Протокол HTTP}
         Будем считать протоколом общепринятый формат общения систем между собой. На сегодняшний день, большинство веб-протоколов работают поверх HTTP. Он относится к прикладным протоколам и работает поверх других протоколов сетевого стека. HTTP — это простой текстовый протокол для передачи любого контента. Изначально разработан для передачи HTML-файлов. Как правило, в веб-разработке используются надстройки поверх протокола. HTTP использует TCP в качестве транспортного протокола, который управляет передачей данных между компьютерами. TCP обеспечивает доставку данных с подтверждением от <<получателя>>. Если <<отправитель>> не получает подтверждения, он повторно отправляет пакет. \cite{http}

         Для передачи данных используются HTTP-запросы и HTTP-ответы. На рисунке 3 приведена структура HTTP-запроса:

         \noindent\begin{minipage}[t]{\textwidth}
             \begin{center}
             \includegraphics[width=0.8 \textwidth]{images/http_request.png}
             \captionof{figure}{Пример HTTP-запроса. Источник: \cite{http}}\end{center} \label{fig:Moreno}
        \end{minipage}
        Здесь,
         \begin{itemize}
             \item HTTP-метод (method) — это глагол, который определяет, какую операцию мы хотим выполнить (GET, POST, PUT и т.д.);
             \item Путь (path) — URL до необходимого нам ресурса;
             \item Версия протокола HTTP (version of the protocol);
             \item Заголовки (headers) — дополнительные параметры запроса, которые нужны серверу;
             \item Опционально: тело запроса — данные, которые мы хотим передать. 
         \end{itemize}
        
        Структура HTTP-ответа показана на рисунке 4. Из нового, здесь: код состояния (status code), указывающий, был ли запрос успешным или нет и почему, а также сообщение о состоянии (status message).

        \noindent\begin{minipage}[t]{\textwidth}
             \begin{center}
             \includegraphics[width=0.8 \textwidth]{images/http_response.png}
             \captionof{figure}{Пример HTTP-ответа. Источник: \cite{http}}\end{center} \label{fig:Moreno}
        \end{minipage}
        

        Теперь, имея базовое представление о HTTP, мы можем перейти к рассмотрению REST, как стиля межсервисного взаимодействия. 

        \subsection{REST} 

        REST (Representational State Transfer, передача репрезентативного
    состояния) – архитектурный стиль построения распределенной системы, в основе которого лежит понятие ресурса и его состояния. Ресурс имеет свой универсальный идентификатор (URI, unified resource identifier), используя который, над данным ресурсом можно совершать различные действия, например,
    CRUD (create, read, update, delete). \cite{rest_soap} \label{REST}

    Важно отметить, что используя REST, нужно думать о приложении с точки зрения ресурсов. Для управления ресурсами, REST использует HTTP-методы. Таким образом, REST – это не столько конкретная конструкция или реализация, сколько архитектурный стиль, позволяющий наилучшим образом использовать функции, предоставляемые HTTP.

    Существуют и другие подходы к микросервисному общению, такие как семейство протоколов RPC, WebSocket и т.д. Однако из-за своей популярности и удобства,  REST-подход остается наиболее распространенным в веб-разработке. В дальнейшем, будет предполагаться использования REST и протокола HTTP для проектирования архитектуры серверной части <<Цифрового двойника сельскохозяйственного поля>>.
        
       \section{Аутентификация и авторизация в микросервисной архитектуре}

        \subsection{Введение}
        Введём определения, необходимые для понимания дальнейшего текста:
        \begin{itemize}
            \item \textit{Аутентификация} - процесс предоставления пользователю доступа к информационной системе. \cite{auth}
             \item \textit{Авторизация} - предоставление определённому лицу или группе лиц прав на выполнение определённых действий. \cite{auth}
        \end{itemize}
        

        Выбор способа аутентификации и авторизации является важной задачей разработчиков, поскольку неправильно реализованная система может стать уязвимостью для хакерских атак и угрозой для конфиденциальности пользовательских данных. В микросервисной архитектуре каждый сервис обычно имеет свою собственную базу данных и собственный сетевой интерфейс, что делает эту задачу особенно сложной. Кроме того, важно обеспечить защищенное соединение между всеми микросервисами, чтобы предотвратить возможность перехвата данных во время передачи. 
        
        Далее будет рассмотрено несколько способов аутентификации в микросервисах.

        
        \subsection{Аутентификация в каждом микросервисе}
        
        \noindent\begin{minipage}[t]{\textwidth}
             \begin{center}
             \includegraphics[width=1 \textwidth]{images/var1.png}
             \captionof{figure}{Вариант с аутентификацией в каждом микросервисе }\end{center} \label{fig:Moreno}
        
        \end{minipage}
        На рисунке 5 показан вариант микросервисной архитектуры с аутентификацией в каждом микросервисе. Так, пользователь, используя клиент-серверный подход, отправляет HTTP-запросы на микросервисы. Для всех микросервисов существует общая база данных (далее БД) <<Users>>, необходимая для получения пользовательских данных для дальнейшей аутентификации и авторизации пользователей. 

        Основной проблемой данного решения является нарушения принципа слабой связанности в микросервисной архитектуре, т.к. все сервисы имеют доступ к одной БД <<Users>>. 

         \subsection{Работа с пользователями в отдельном микросервисе. Реализация 1}
    
         В данной реализации, существует микросервис аутентификации, на который каждый из микросервисов делает запрос (см. рисунок 6). 
         \noindent\begin{minipage}[t]{\textwidth}
             \begin{center}
             \includegraphics[width=1 \textwidth]{images/var2.png}
             \captionof{figure}{Работа с пользователями в отдельном микросервисе. Реализация 1}\end{center} \label{fig:Moreno}
        
        \end{minipage}
         Недостатком такого решения можно считать реализацию логики обращения к микросервису аутентификации для каждого микросервиса. Также стоит отметить, что при такой реализации, микросервисы не знают, какие методы требуют аутентификации. Следовательно, на каждый запрос, отправленный в  $microservice$ $i$, $microservice$ $i$ будет отправлять запрос в микросервис аутентификации для проверки наличия доступа.

         \subsection{Работа с пользователями в отдельном микросервисе. Реализация 2}
    
         Удобство данного метода заключается в том, чтобы клиентской части приложения уже не нужно знать о том, к какому из микросервисов надо обращаться, так как все запросы проходят через микросервис <<Auth>> (см. рисунок 7). Также решены проблемы предыдущей реализации.
         \noindent\begin{minipage}[t]{\textwidth}
             \begin{center}
             \includegraphics[width=1 \textwidth]{images/var3.png}
             \captionof{figure}{Работа с пользователями в отдельном микросервисе. Реализация 2}\end{center} \label{fig:Moreno}
        
        \end{minipage}
        
        Хотя на первый взгляд это решение может показаться оптимальным, оно также обладает недостатками. Во-первых, теперь все запросы проходят через микросервис <<Auth>>, даже те, что не требуют аутентификации. Как следствие, некоторые запросы проходят <<неоптимальный маршрут>>. Во-вторых, микросервис аутентификации имеет свою базу данных, тяжеловесные компоненты (фреймворк, утилиты и т.д.), а значит при каждом запросе, даже не требующем аутентификации, происходит инициализация ядра сервиса, что замедляет время работы всего приложения. 
    
        \subsection{Паттерн API Gateway} \label{api_gateway}
        API Gateway - это прокси-сервер, который предоставляет единую точку входа для клиентов и обеспечивает управление и контроль доступа к различным микросервисам и их API. API Gateway должен иметь возможность определять, к какому микросервису направить запрос клиента на основе заданных правил и путей. В этом случае, реализация серверной части приложения будет скрыта от клиента, так, обеспечиваются принцип разделения обязанностей (Separation of Concerns) и дополнительная безопасность. Важно отметить, что API Gateway должен быть максимально быстрым, так как все запросы идут через него. \cite{habr}

        Вариант с API Gateway отображен на рисунке 8:

        \noindent\begin{minipage}[t]{\textwidth}
                 \begin{center}
                 \includegraphics[width=1 \textwidth]{images/var4.png}
                 \captionof{figure}{API Gateway}\end{center} \label{fig:Moreno}
            
            \end{minipage}

        
   

    \noindent\begin{minipage}[t]{\textwidth}
                 \begin{center}
                 \includegraphics[width=1 \textwidth]{images/var5.png}
                 \captionof{figure}{Аутентификация с API Gateway}\end{center} \label{fig:Moreno}
            
            \end{minipage}

             В таком случае процесс аутентификации можно описать так (cм. рисунок 9): 
    \begin{enumerate}
        \item Пользователь вводит логин и пароль;
        \item Информация приходит на API Gateway;
        \item API Gateway видит путь \textit{api/v1/auth} и отправляет пришедшие ему данные на микросервис <<Auth>>;
        \item Микросервис <<Auth>> проверяет пользователя, выдает токен и передает его в API Gateway;
        \item API Gateway передает токен пользователю;
        
    \end{enumerate}


    \subsection{Авторизация и JWT}

    Для пользования нашим приложением, клиенты должны аутентифицировать свои учетные данные. Однако данный механизм никак не разделяет аутентифицированных пользователей по правам доступа к ресурсам. Для этого необходимо реализовать механизм авторизации.

    JSON Web Token (JWT) — это открытый стандарт, определяющий компактный и автономный способ безопасной передачи информации между сторонами в виде объекта JSON. Безопасность подтверждается цифровой подписью. \cite{jwt}

    В основе JWT лежат:

    \begin{itemize}
        \item Заголовок (header) — информация про алгоритм шифрования;
         \item Полезная нагрузка (payload) — данные, которые хотим передать от одного микросервиса к другому; 
        \item Сигнатура (signature) — то, за счет чего обеспечивается безопасность. Информация шифруется по выбранному алгоритму хеширования через секретный ключ, заранее известному всем микросервисам, которым нужна авторизация.
    \end{itemize}

    \noindent\begin{minipage}[t]{\textwidth}
                 \begin{center}
                 \includegraphics[width=1 \textwidth]{images/var6.png}
                 \captionof{figure}{Пример JWT в закодированном и декодированном видах. Источник: \cite{jwt}}\end{center} \label{fig:Moreno}
            
            \end{minipage}


    
    JWT передается в заголовках HTTP-запросов в зашифрованном (encoded) виде - в виде обычной строки (см. рисунок 10). <<Получатель>> сможет декодировать (decode) JWT для получения полезной информации при наличии секретного ключа.


    \section{Итоговая архитектура}

    Для создания архитектуры, нам осталось только определить необходимые сервисы (результат главы \ref{pattern}). Так как они являются REST API (результат главы \ref{messaging}), то каждый микросервис должен представлять собой ресурс. 
    Из функциональных требований (глава \ref{functional}) и выбранного архитектурного паттерна (\ref{api_gateway}), формирируем следующий список компонентов:
    \begin{enumerate}
        \item API Gateway;
        \item Микросервис полей;
        \item Микросервис аутентификации;
        \item Микросервис профилей;
        \item Микросервис метеоданных, являющийся также ETL-системой;
    \end{enumerate}
    
    
    Получившаяся архитектура отображена на рисунке 11:

     \noindent\begin{minipage}[t]{\textwidth}
                 \begin{center}
                 \includegraphics[width=1 \textwidth]{images/new_arch.png}
                 \captionof{figure}{Архитектура серверной части веб-приложения <<Цифровой двойник сельскохозяйственного поля>>.}\end{center} \label{fig:Moreno}
            
            \end{minipage}

    ER-диаграмма распределенной базы данных отображена на рисунке 12:

    \noindent\begin{minipage}[t]{\textwidth}
                 \begin{center}
                 \includegraphics[width=1 \textwidth]{images/diagram-2.png}
                 \captionof{figure}{ER-диаграмма распределенной базы данных веб-приложения <<Цифровой двойник сельскохозяйственного поля>>.}\end{center} \label{fig:Moreno}
            
            \end{minipage}
    

    Следующим шагом будет выбор как общих технологий, так и конкретных для реализации соответствующих сервисов. 

\chapter{Программная разработка серверной части приложения}
    \section{Выбор технологий}
    В качестве основного языка программирования для разработки микросервисов был выбран Python. Python является одним из наиболее популярных языков программирования для веб-разработки. Есть несколько причин, почему Python является оптимальным вариантом:

    \begin{enumerate}
        \item Простота и удобство использования: Python имеет простой и понятный синтаксис, который легко читать и писать. Это делает его идеальным языком для написания микросервисов, основной нагрузкой в которых являются I/O-bound задачи.
        \item Большое количество библиотек и фреймворков: Python имеет огромное сообщество разработчиков, которые создали множество библиотек и фреймворков, которые упрощают и ускоряют процесс разработки веб-приложений. Некоторые из наиболее популярных фреймворков для веб-разработки на Python: Django, FastAPI, Flask.
        \item Большое количество инструментов для тестирования: Python имеет множество инструментов для тестирования, которые помогают разработчикам создавать качественное и надежное программное обеспечение.
    \end{enumerate}

    В качестве фреймворка был выбран FastAPI \cite{fastapi} – веб-фреймворк для реализации REST API. Одним из его преимуществ является поддерживаемая асинхронность “из коробки”, которая так важна для быстрого выполнения I/O-bound задач, которыми в свою очередь являются основные функции сервисов аутентификации и профилей пользователей. Также важно упомянуть о поддерживаемой валидации данных в FastAPI, достигаемой использованием библиотеки Pydantic. \cite{pydantic}
    
    В качестве СУБД был выбран PostgreSQL.
    
    \section{Микросервис аутентификации}
    Одной из основных задач при разработке данного сервиса являлось составление структуры JWT:
    \begin{verbatim}
    class TokenPayloadSchema(BaseModel):
        iat: datetime
        exp: datetime
        sub: str
        role: str
        email: str
        org: int
    \end{verbatim}

    Так, помимо передачи стандартных полей, таких как $iat$, $exp$ и $sub$, мы дополнительно передаем роль пользователя - для авторизации в каждом сервисе, адрес его электронной почты - для корректной работы email-сервиса, а также id его организации (для организации значения $sub$ и $org$ будут совпадать). 

    Для обеспечения безопасности пользовательских данных, пароли хранятся в зашифрованном с помощью алгоритма хэширования bcrypt \cite{bcrypt} формате. Таким образом, даже при получении злоумышленниками доступа к базе даннных с пользовательскими данными, пароли не смогут быть расшифрованы. 

    Сервис был разработан на базе фреймворка FastAPI с использованием СУБД PostgreSQL. В последствии сервис был обёрнут в docker-контейнер. 
    
    На рисунке 13 представлен API-контракт разработанного сервиса:
        \noindent\begin{minipage}[t]{\textwidth}
                 \begin{center}
                 \includegraphics[width=1 \textwidth]{images/auth_contract.png}
                 \captionof{figure}{API-контракт сервиса аутентификации, сгенерированный инструментом Swagger.}\end{center} \label{fig:Moreno}
            
            \end{minipage}
            
    Ссылка на исходный код: \url{https://github.com/AgroScience-Team/auth-service}.


    \section{Микросервис метеоданных}
    \subsection{Выбор внешней системы для получения данных}

     Из требований \ref{functional} имеем следующий списоок необходимых метеоданных для получения из внешних систем:

    \begin{itemize}
        \item Текущая температура;
        \item Текущая влажность;
        \item Текущий уровень осадков;
        \item Текущая точка росы;
        \item Текущая скорость ветра;
        \item Текущая температура почвы на глубине/глубинах 0-18 см;
        \item Текущая влажность почвы на глубине/глубинах от 0-27 см;
        \item Время восхода солнца;
        \item Время заката;
        \item Сумма осадков за сутки;
    \end{itemize}

    На сегодняшний день существует множество внешних систем, предоставляющих API для получения метеоданных. Рассмотрим популярные варианты.

    \begin{enumerate}
        \item \textit{OpenWeather} \cite{openweather} - онлайн сервис, который предоставляет платный (есть функционально ограниченная бесплатная версия) API для доступа к данным о текущей погоде, прогнозам и историческим данным.  
        
        \textbf{Плюсы}:  
        \begin{itemize}
                \item Широкий функционал в платной версии сервиса;
                \item Возможность получения данных в разных форматах; 
        \end{itemize}
        
        \textbf{Минусы}:
        \begin{itemize}
                \item API не работает на территории РФ;
                \item Ограниченное число запросов к API в бесплатной версии: 1000 в день. Доступны разные тарифы;
                \item Нет данных о влажности и температуре почве в бесплатной версии;
        \end{itemize}

        \item \textit{API Яндекс Погоды} \cite{yandex_weather} -  API, предоставляющее погодные и климатические данные. Передаёт текущие данные и прогноз по 150+ погодным параметрам: от базовых (температура, осадки) до нишевых, которые важны для конкретного бизнеса.  
        
        \textbf{Плюсы}:  
        \begin{itemize}
                \item Широкий функционал сервиса;
                \item Возможность получения данных в разных форматах; 
                \item Быстрое время ответа: до 500мс в 95 процентах от всех запросов;
        \end{itemize}
        
        \textbf{Минусы}:
        \begin{itemize}
                \item Отсутствие бесплатной версии: от 54.000 рублей за месяц использования. Доступны разные тарифы;
        \end{itemize}

        \item \textit{Open-Meteo} \cite{open_meteo} -  API погоды с бесплатным доступом для некоммерческого использования.
        
        \textbf{Плюсы}:  
        \begin{itemize}
                \item Широкий функционал и гибкая настройка: есть возможность получить все необходимые метеоданные;
                \item Цена: бесплатно;
                \item Нет ограничения на количество запросов; 
                \item Возможность получения данных в разных форматах; 
        \end{itemize}
        
        \textbf{Минусы}:
        \begin{itemize}
                \item Для решения поставленной задачи недостатков выявлено не было;
        \end{itemize}
        
    \end{enumerate}

    Таким образом, Open-Meteo был выбран источником для загрузки метеоданных.

    \subsection{Реализация сервиса}
        Сервис был разработан на базе фреймворка FastAPI с использованием СУБД PostgreSQL. В последствии сервис был поднят в docker-контейнере. 
        
        Для парсинга метеоданных использовалась библиотека HTTPX \cite{HTTPX} - полнофункциональный HTTP-клиент для Python 3, который предоставляет синхронные и асинхронные API, а также поддерживает протоколы HTTP/1.1 и HTTP/2.

        В качестве библиотеки для реализации планировщика задач был выбран ApScheduler \cite{apscheduler}. Данной библиотекой легко пользоваться, а также она не требует подключения дополнительных компонентов, таких как очередь задач и результирующее хранилище.

        По итогу сервис представляет из себя одновременно REST API и ETL-систему.
        На рисунке 14 представлен API-контракт разработанного сервиса:
        \noindent\begin{minipage}[t]{\textwidth}
                 \begin{center}
                 \includegraphics[width=1 \textwidth]{images/meteo_contract.png}
                 \captionof{figure}{API-контракт сервиса метеоданных, сгенерированный инструментом Swagger.}\end{center} \label{fig:Moreno}
            
            \end{minipage}
            
    Ссылка на исходный код: \url{https://github.com/AgroScience-Team/meteo-service}.


    
    \section{Микросервис профилей}

    Сервис был разработан на базе фреймворка FastAPI с использованием СУБД PostgreSQL. В последствии сервис был поднят в docker-контейнере. 
    
    На рисунке 15 представлен API-контракт разработанного сервиса:
        \noindent\begin{minipage}[t]{\textwidth}
                 \begin{center}
                 \includegraphics[width=1 \textwidth]{images/profiles_contract.png}
                 \captionof{figure}{API-контракт сервиса аутентификации, сгенерированный инструментом Swagger.}\end{center} \label{fig:Moreno}
            
            \end{minipage}
            
    Ссылка на исходный код: \url{https://github.com/AgroScience-Team/profiles-service}.
    
    \section{API Gateway}
    
    Для реализации микросервисного паттерна API-Gateway был выбран Nginx. Главной причиной для выбора Nginx являлась простота его настройки в качестве обратного прокси-сервера \cite{nginx}.
    
    При каждом пользовательском запросе, подразумевающем обращение к сервисам fields-service, profiles-service и meteo-service, nginx отправлял подзапрос \textit{api/auth/introspect} в сервис аутентификации для валидации отправленного токена в заголовке первоначального запроса. Если отправленный токен проходил валидацию успешно, то дальше отправлялся первоначальный пользовательский запрос в соответствующий микросервис.

    Для поднятия api-gateway необходимо запустить все docker-контейнеры остальных сервисов в одной docker-сети. При этом каждый сервис должен использовать уникальный порт.
    
    Ссылка на исходный код: \url{https://github.com/AgroScience-Team/api-gateway}.
    
\chapter{Тестирование и развёртывание на удалённой машине}
Во время разработки проекта писались unit-тесты с помощью библиотеки Pytest \cite{pytest}. Это помогло выявить и устранить ошибки на ранних этапах разработки. Кроме того, выполнялось ручное тестирование с помощью инструментов Swagger \cite{swagger} и Postman \cite{postman}. Это позволило проверить правильность работы API, а также ручные сценарии использования приложения. Благодаря комбинации автоматизированных и ручных тестов было значительно повышено качество приложения и появилась уверенность в его надежности перед развёртыванием на сервере.

Для развертывания приложения на удаленной машине использовался протокол SSH для безопасного соединения и управления сервером. После установки необходимых зависимостей и настройки конфигурации был произведен запуск приложения.

Сейчас доступ к приложению можно получить по ссылке \url{http://www.agromelio.ru}. Пользователи могут обращаться к приложению просто перейдя по данной ссылке и пользоваться его функционалом без необходимости устанавливать какие-либо дополнительные программы.


\chapter{Заключение}
    \section{Результаты работы}

    В современном мире происходит активная интеграция информационнных технологий в сельское хозяйство. Широкое применение находят приложения для точного земледелия, с помощью которых ведется хранение различной информации о сельхозугодьях в понятном для человека виде. 

   В ходе выполнения поставленных задач из \ref{target} были выполнены следующие подзадачи:

\begin{enumerate}
    \item Обзор исследований и существующих решений:
    
    \begin{itemize}
    \item Проведено исследование современного состояние проблемы использования информационных технологий в сельском хозяйстве;
    \item Проведен анализ современных программных решений для точного земледелия;
    \end{itemize}

    \item Формирование требований к приложению:
    \begin{itemize}
    \item Были сформированы функциональные и нефункциональные требования к приложению;
    \end{itemize}

    \item Проектирование архитектуры серверной части приложения:
    \begin{itemize}
    \item Изучены и выбраны оптимальные паттерны проектирования серверной части приложения. Выбор микросервисного паттерна позволил выполнить нефункциональное требование \ref{non_functional} о расширяемости системы.
    \item Изучены различные способы межсервисного взаимодействия, выбран оптимальный вариант для данной задачи - REST API;
    \item Были исследованы различные способы аутентификации и авторизации в микросервисных приложениях, был выбран оптимальный вариант - проксирование запросов при помощи API Gateway с авторизацией в каждом сервисе;
    \item Разработана схема итоговой архитектуры серверной части приложения;
    \item Разработана ER-диаграмма распределенной базы данных;
    \end{itemize}

    \item Программная разработка серверной части приложения:
    \begin{itemize}
    \item Выбраны общие технологии для разработки бекенда приложения, а также технологии для реализации точечных подзадач;
    \item Разработаны микросервисы аутентификации, профилей, метеоданных и API Gateway;
    \item Для микросервиса аутентификации был выбран алгоритм шифрования пользовательских паролей для безопасного хранения. Также была разработана схема JWT.
    \item Для микросервиса метеоданных проведено исследование внешних систем для сбора данных. Выбран оптимальный вариант. 
    \end{itemize}

    \item Тестирование и развертывание приложения на удалённом сервере:
    \begin{itemize}
    \item Написаны unit-тесты для отдельных функций сервисов;
    \item Проведено ручное тестирование с помощью инструментов Swagger и Postman;
    \item Все микросервисы были подняты в docker-контейнерах в рамках одной docker-сети и объединены сервисом api-gateway.
    \item Приложение было развернуто на удалённой машине.
    \end{itemize}
    

    \end{enumerate}
    



\section{Дальнейшее развитие}

Благодаря выбранному архитектурному решению, дальнейшее развитие сервиса может ограничивается фантазией заказчика или разработчика.

Можно рассмотреть добавление функционала по хранению и обработке ортофотопланов. Ортофотопланы представляют собой изображения, полученные в результате аэрофотосъемки и прошедшие специальную обработку для исключения искажений. Эти данные могут быть ценными для сельскохозяйственных предприятий, позволяя им более эффективно планировать работу на полях, определять места повреждения растений или оценивать объем урожая.

Следующим шагом может быть внедрение сервиса рассылки сообщений, который позволит агрономам и фермерам оперативно уведомляться о важных событиях и новостях, касающихся их деятельности. Такие уведомления могут касаться изменения погодных условий или напоминаний о скором конце срока текущего посева.

Кроме того, для улучшения аналитики и прогнозирования процессов на полях, можно добавить функцию получения метеоданных за выбранный период. Это позволит агрономам анализировать погодные условия в разные периоды времени, выявлять тенденции и коррелировать их с результатами урожайности. Автоматическое построение графиков сделает этот процесс более удобным и эффективным, освобождая время специалистов для других задач и принятия важных решений.